\documentclass{resume} % Use the custom resume.cls style

%\usepackage[left=0.4 in,top=0.4in,right=0.4 in,bottom=0.4in]{geometry} % Document margins
\usepackage[left=0.75in,top=0.6in,right=0.75in,bottom=0.6in]{geometry} % Document margins
\newcommand{\tab}[1]{\hspace{.2667\textwidth}\rlap{#1}} 
\newcommand{\itab}[1]{\hspace{0em}\rlap{#1}}


\name{Indrajit Ghosh} % Your name
% You can merge both of these into a single line, if you do not have a website.
\address{Stat-Math Unit,  Indian Statistical Institute \\ 8th Mile, Mysore Road \\ RVCE Post \\ Bangalore 560 059} 
\address{\href{mailto:rs_math1902@isibang.ac.in}{\texttt{rs\_math1902@isibang.ac.in}} \\  \href{mailto:indrajitghosh912@gmail.com}{\texttt{indrajitghosh912@gmail.com}}}  %


%%%%%%%%%%%%%%%% PDF Setups %%%%%%%%%%%
\usepackage{hyperref}
\hypersetup{
	pdftitle={Curriculum Vitae},
	pdfauthor={Indrajit Ghosh},
	pdfsubject={CV},
	pdfcreationdate={\today},
	pdfcreator={LaTeX},
	pdfkeywords={Curriculum Vitae},
	colorlinks=true,
	pdfpagemode=UseOutlines,
}

\newcommand{\weblink}[1]{\texttt{\href{#1}{#1}}}



\begin{document}

%%%%%%%%%%%%%%%%%%%%%%%%%%%%
%%%% Education
%%%%%%%%%%%%%%%%%%%%%%%%%%%%

\begin{rSection}{Education}
{\bf Philosophiae Doctorate in Mathematics},  \hfill {\em Jul 2019 - Present}\\ {\it Stat-Math Unit, Indian Statistical Institute Bangalore} \hfill {\em Bangalore, India}

{\bf Master of Science in Pure Mathematics},  \hfill {\em 2016 - 2018}\\ {\it Department of Pure Mathematics, University of Calcutta} \hfill {\em Kolkata, India}

{\bf Bachelor of Science in Mathematics (Hons.)},  \hfill {\em 2013 - 2016}\\ {\it Barasat Government College} \hfill {\em Barasat, India}

\end{rSection}


\begin{rSection}{Research INTERESTs}
	Operator Algebras: von-Neumann Algebras, Type $II_1$ Factors, $C^*$-algebras. Operator Theory, Topology.
	
\end{rSection}


\begin{rSection}{FELLOWSHIPS AND ACHIEVEMENTS}
	\begin{rSubsection}{}{}{}{}
		\item Selected for \textbf{NBHM JRF} fellowship, \emph{National Board for Higher Mathematics}, Dept. Of Atomic Energy, Govt. of India, 2019, Ref No. 0203/11/2019-R\&D-II/9261.
		
		\item Selected as JRF at Indian Statistical Institute, 2019
		
		\item Selected as JRF at IIT Madras, 2019
		
		\item Selected as JRF at IISER Bhopal, 2019
		
		\item Qualified \textbf{GATE} in Mathematics(MA), 2019
		
		\item Qualified Joint \textbf{CSIR-UGC NET JRF} fellowship, Dec 2018
	\end{rSubsection}

\end{rSection}


\begin{rSection}{PROJECTS \& OTHER ACTIVITIES}
	
	\begin{rSubsection}{Directed Reading Project}{Jan 2021 - May 2021}{Indian Statistical Institute}{Bangalore, India}
		\item Supervisor: Prof. Soumyashant Nayak
		\item Topics studied: $C^*$-Algebras, States and Representations, SOT and WOT on $\mathcal{B}(\mathcal{H})$, von-Neumann Algebras.
	\end{rSubsection}

	\begin{rSubsection}{Graduate Course Work: Operator Theory}{Sep 2020}{Indian Statistical Institute}{Bangalore, India}
		\item Instructor: Prof. Rajaram Bhat
		\item Topics Studied: Locally Convex Spaces, Operators on Hilbert Spaces, Banach Algebras, Basic $C^*$-Algebra Theory.
	\end{rSubsection}

\end{rSection}


\begin{rSection}{CONFERENCES AND SCHOOLS ATTENDED}
	
	\begin{rSubsection}{Baby Steps Beyond the Horizon - school for students}{Aug 29-Sep 02, 2022}{Institute of Mathematics Polish Academy of Science}{Warszawa, Poland}
		\item Place: Online Mode
		\item Website: \weblink{https://www.impan.pl/en/activities/banach-center/conferences/22-babysteps}
	\end{rSubsection}
	
	\begin{rSubsection}{A conference on ergodic theory and von Neumann algebra}{Aug 04-06, 2022}{National Institute of Science Education and Research}{Odisha, India}
		\item Place: Online mode
		\item Website: \weblink{https://www.niser.ac.in/etvna/}
	\end{rSubsection}

	\begin{rSubsection}{IFCAM Summer School - Mathematical Aspects of Quantum Mechanics}{Jun 1-12, 2022}{Indian Institute of Science}{Bangalore, India}
		\item Place: Indian Institute of Science, Bangalore
		\item Website: \weblink{https://ifcam.sciencesconf.org/}
	\end{rSubsection}
	
	\begin{rSubsection}{Advanced Instructional School - Advanced Linear Algebra}{May 2022}{Indian Statistical Institute}{Bangalore, India}
		\item Organisers: Prof. Jaydeb Sarkar (Indian Statistical Institute, Bangalore), Prof. Rajesh Sharma (Himachal Pradesh University)
		
		\item Website: \weblink{https://www.atmschools.org/school/2022/AIS/ala}
	\end{rSubsection}

	\begin{rSubsection}{Conference on Quantum Probability and Infinite Dimensional Analysis}{Jan 17-20, 2022}{Indian Statistical Institute}{Bangalore, India}
		\item Local organisers: Prof. Rajaram Bhat (Indian Statistical Institute, Bangalore), Prof. Jaydeb Sarkar (Indian Statistical Institute, Bangalore)
		
		\item Website: \weblink{https://www.isibang.ac.in/~statmath/conferences/QPIDA-2022.html}
	\end{rSubsection}

	\begin{rSubsection}{IWM Mini Course on Approximate Solutions of Operator Equations and Eigenvalue Problems}{Nov 2021}{}{}
		\item Place: Online mode
		\item Organisers: \href{https://sites.google.com/site/iwmmath/}{Indian Women and Mathematics}
		
		\item Website: \href{https://drive.google.com/file/d/1FYd-6j_hwOMfc2Z9NIQiv2ZqSnF859J4/view}{IWM-MiniCourse-2021}
	\end{rSubsection}

	\begin{rSubsection}{Online Workshop on Algebraic Number Theory}{Sep 2020}{Assam University}{Silchar, India}
		\item Place: Online mode
		\item Organiser: Prof. \href{sbsap.aus@gmail.com}{Samira Behera} (Assam University, Silchar)
	\end{rSubsection}

	
\end{rSection}

\begin{rSection}{TEACHING EXPERIENCE}
	
	\begin{rSubsection}{Teaching Assistant}{Aug 2022 - Present}{Indian Statistical Institute}{Bangalore, India}
		\item Course: Complex Analysis I (BMath III Year)
		\item Instructor: Mathew Joseph
	\end{rSubsection}

	\begin{rSubsection}{Teaching Assistant}{Jan 2022 - May 2022}{Indian Statistical Institute}{Bangalore, India}
		\item Course: Functional Analysis I (MMath I Year)
		\item Instructor: Soumyashant Nayak
	\end{rSubsection}
	
	\begin{rSubsection}{Teaching Assistant}{Sep 2021 - Jan 2021}{Indian Statistical Institute}{Bangalore, India}
		\item Course: Optimization (BMath II Year)
		\item Instructor: Soumyashant Nayak
	\end{rSubsection}

\end{rSection}



\end{document}
